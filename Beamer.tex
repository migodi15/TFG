\documentclass{beamer}
\usetheme{UASLP1}
\usepackage[utf8]{inputenc}
\usepackage[T1]{fontenc}
\usepackage[catalan]{babel}
%% Use any fonts you like.
\usepackage{helvet}
\usepackage{yhmath}

\title{Els Teoremes Petit i Gran de Picard}
\subtitle{Treball Final del Grau de Matemàtiques}
\author{Mireia Gómez Diaz}
\date{12 de juliol de 2021}
%\institute{\url{m.gomezi@e-campus.uab.cat}}

\newtheorem{defi}{Def.}[section]
\newtheorem{teorema}{Teorema}[section]
\newtheorem{corollari}{Corol$\cdot$lari}[teorema]

\begin{document}

\begin{frame}[plain,t]
\titlepage
\end{frame}

%%%%%%%%%%%%%%%%%%%%%%%%%%%%%%%%%%%%%%%%%%%%%%%%%%%%%%%%%%%%%%%%%%%%

\section{Exposició}
\begin{frame}
\frametitle{Singularitat essencial}

\setbeamercolor{block title}{use = structure, fg=violet!60!black, bg=violet!25!white}
\setbeamercolor{block body}{use=structure, fg=black, bg=violet!10!white} 

\begin{defi}[singularitat essencial]
Direm que $z_0$ és una singularitat essencial de f(z) si $z_0$ és una singularitat aïllada i $z_0$ no és ni un pol ni una singularitat evitable. 
\end{defi}

\end{frame}

%%%%%%%%%%%%%%%%%%%%%%%%%%%%%%%%%%%%%%%%%%%%%%%%%%%%%%%%%%%%%%%%%%%%

\subsection{Motivació}
\begin{frame}
\frametitle{Motivació}

\begin{teorema}[de Casoratti-Weierstrass]
Sigui $f$ una funció holomorfa amb una singularitat essencial en $z=a$. Llavors, en cada entorn d'$a$, la imatge d'$f(z)$ és densa en $\mathbb{C}$.
\end{teorema}

\end{frame}

%%%%%%%%%%%%%%%%%%%%%%%%%%%%%%%%%%%%%%%%%%%%%%%%%%%%%%%%%%%%%%%%%%%%

\subsection{Teorema Gran de Picard}
\begin{frame}
\frametitle{Teorema Gran de Picard}

\begin{teorema}[Teorema Gran de Picard]
Sigui $f$ una funció holomorfa amb una singularitat essencial en $z=a$. Llavors, en cada entorn d'$a$, $f(z)$ pren tots els valors complexos possibles infinites vegades amb, com a màxim, l'excepció d'un punt.
\end{teorema}

\begin{exampleblock}{Exemple}
\begin{center}
    $f(z)=e^{1/z}$
\end{center}
\end{exampleblock}

\end{frame}

%%%%%%%%%%%%%%%%%%%%%%%%%%%%%%%%%%%%%%%%%%%%%%%%%%%%%%%%%%%%%%%%%%%%

\subsection{Teorema Petit de Picard}
\begin{frame}
\frametitle{Teorema Petit de Picard}
\begin{teorema}[Teorema Petit de Picard]
Sigui $f$ una funció entera que omet dos valors, llavors $f$ és constant.
\end{teorema}

\begin{exampleblock}{Exemple}
\begin{center}
    $f(z)=e^z$
\end{center}
\end{exampleblock}

\end{frame}

%%%%%%%%%%%%%%%%%%%%%%%%%%%%%%%%%%%%%%%%%%%%%%%%%%%%%%%%%%%%%%%%%%%%

\subsection{Objectius}
\begin{frame}
\frametitle{Objectius}

L'objectiu d'aquest treball ha estat entendre i demostrar els Teoremes Petit i Gran de Picard. 
\vspace{5mm}

Per aconseguir-ho, hem fet servir eines com:
 
 - El Teorema de Bloch.
 
 - El Teorema de Schottky.

 - Els Teoremes de Montel.


\end{frame}

%%%%%%%%%%%%%%%%%%%%%%%%%%%%%%%%%%%%%%%%%%%%%%%%%%%%%%%%%%%%%%%%%%%%

\subsection{Teorema de Bloch}
\begin{frame}
\frametitle{Teorema de Bloch}

\begin{teorema}[de Bloch]
Sigui $f$ una funció analítica en un domini que conté el disc tancat $\overline{D(0,1)}$ i que satisfà $f(0)=0$ i $f'(0)=1$. Llavors existeix un disc $S \subset D(0,1)$ on $f$ és injectiva i $f(S)$ conté un disc de radi $1/72$.
\end{teorema}

Nota: $1/72 = 0,013\wideparen{8}$.

\end{frame}

%%%%%%%%%%%%%%%%%%%%%%%%%%%%%%%%%%%%%%%%%%%%%%%%%%%%%%%%%%%%%%%%%%%%

\begin{frame}
\frametitle{Constant de Bloch}

\setbeamercolor{block title}{use = structure, fg=violet!60!black, bg=violet!25!white}
\setbeamercolor{block body}{use=structure, fg=black, bg=violet!10!white} 

\begin{defi}[constant de Bloch]
Considerem $\mathcal{F}$ el conjunt de funcions analítiques en un domini que conté el disc tancat $\overline{D(0,1)}$ i que satisfan $f(0)=0$, $f'(0)=1$. Per cada $f \in \mathcal{F}$ considerem $\beta (f)$ el suprem de tots els nombres $r$ per als quals hi ha un disc $S\subset D(0,1)$ en que $f$ és injectiva i $f(S)$ conté un disc de radi $r$. Definim la constant de Bloch com 
$$B=\inf \{\beta (f): f\in \mathcal{F}\}.$$
\end{defi}

Nota: $0,43\leq B \leq 0,48$.
\end{frame}


%%%%%%%%%%%%%%%%%%%%%%%%%%%%%%%%%%%%%%%%%%%%%%%%%%%%%%%%%%%%%%%%%%%%

\subsection{Teorema de Schottky}

\begin{frame}
\frametitle{Teorema de Schottky}
\begin{teorema}[de Schottky]
Per cada $\alpha$ i $\beta$, $0<\alpha<\infty$ i $0\leq\beta\leq1$, existeix una constant $C(\alpha,\beta)$ tal que si $f$ és una funció analítica en un domini que conté el disc tancat $\overline{D(0,1)}$ i que no pren els valors $0$ ni $1$ i $|f(0)|\leq\alpha$, llavors $|f(z)|\leq C(\alpha,\beta)$ per a tot $|z|\leq\beta$.


\end{teorema}

\end{frame}

%%%%%%%%%%%%%%%%%%%%%%%%%%%%%%%%%%%%%%%%%%%%%%%%%%%%%%%%%%%%%%%%%%%%
\subsection{Teoremes de Montel}
\begin{frame}
\frametitle{Normalitat i equicontinuïtat}

\setbeamercolor{block title}{use = structure, fg=violet!60!black, bg=violet!25!white}
\setbeamercolor{block body}{use=structure, fg=black, bg=violet!10!white} 

\begin{defi}[família normal]
Sigui $\mathcal{F}$ una família de funcions definida sobre un conjunt obert del pla complex. Direm que $\mathcal{F}$ és normal si tota successió de $\mathcal{F}$ té una subsuccessió que convergeix uniformement sobre els compactes de l'obert.
\end{defi}

\begin{defi}[família equicontínua]
Sigui $\mathcal{F} \subset C(G,\Omega)$. Direm que $\mathcal{F}$ és equicontínua en un punt $z_0$ si per a tot $\varepsilon>0$ existeix $\delta>0$ tal que si $|z-z_0|<\delta$ aleshores la distància entre $f(z)$ i $f(z_0)$ és menor a $\varepsilon$ per a tot $f \in \mathcal{F}$.
\end{defi}

\end{frame}

%%%%%%%%%%%%%%%%%%%%%%%%%%%%%%%%%%%%%%%%%%%%%%%%%%%%%%%%%%%%%%%%%%%%

\begin{frame}
\frametitle{Teoremes de Montel}

\begin{teorema}[de Montel]
Una família de funcions holomorfes definides en un conjunt obert del pla complex és normal si, i només si, està uniformement acotada en els compactes de l'obert.
\end{teorema}

\begin{teorema}[de Montel-Caratheodory]
Sigui $\mathcal{F}$ la família de funcions analítiques d'un domini $\Omega$ que no prenen els valors $0$ ni $1$. Llavors $\mathcal{F}$ és normal a $C(\Omega,\mathbb{C}_\infty)$.
\end{teorema}

\end{frame}


%%%%%%%%%%%%%%%%%%%%%%%%%%%%%%%%%%%%%%%%%%%%%%%%%%%%%%%%%%%%%%%%%%%%

\begin{frame}
\frametitle{Resultats que se'n deriven}


\begin{corollari}[I]
Si $f$ té una singularitat aïllada a $z=a$ i hi ha dos nombres complexos que $f$ no pren infinitament sovint, llavors $z=a$ és un pol o una singularitat evitable.
\end{corollari}

\begin{corollari}[II]
Si $f$ és una funció entera no polinomial, llavors $f$ pren tots els valors complexos un nombre infinit de vegades excepte, com a molt, un valor.
\end{corollari}

\end{frame}


%%%%%%%%%%%%%%%%%%%%%%%%%%%%%%%%%%%%%%%%%%%%%%%%%%%%%%%%%%%%%%%%%%%%

\begin{frame}
\Large{\centerline{Moltes gràcies!}}
\end{frame}

\end{document}